En los últimos años, el crecimiento exponencial de los datos, impulsado por la digitalización de procesos y la expansión de servicios en línea, ha planteado nuevos retos en el ámbito del almacenamiento, recuperación y gestión de información. Frente a este escenario, las bases de datos tradicionales del tipo relacional, o SQL, han demostrado ciertas limitaciones en cuanto a escalabilidad, flexibilidad y rendimiento cuando se trata de manejar grandes volúmenes de datos. Como respuesta, surgieron las bases de datos NoSQL, las cuales ofrecen una alternativa viable y eficiente para aplicaciones distribuidas y con altos requerimientos de rendimiento, especialmente en entornos orientados al Big Data, la analítica en tiempo real y la web semántica.

Las bases de datos NoSQL se caracterizan por su capacidad para escalar horizontalmente, su modelo flexible de datos, y su especialización en tipos particulares de almacenamiento como clave-valor, documentos, grafos, columnas anchas, etc. Ahora bien, tanta variedad tiene su lado negativo, resulta difícil adoptar estas tecnologías cuando las aplicaciones existentes dependen por completo de SQL para consultar, manipular y definir datos.

El problema de fondo está en que SQL y NoSQL hablan idiomas distintos. SQL es un lenguaje declarativo, maduro y con décadas de estandarización detrás, en cambio, cada sistema NoSQL trae consigo su propia sintaxis y manera de operar. Para un desarrollador acostumbrado a escribir JOINs y WHERE, enfrentarse a APIs propietarias supone una curva de aprendizaje considerable. Y si hablamos de migrar un sistema legado o de hacer que dos bases de datos colaboren, los dolores de cabeza se multiplican.

Han surgido alternativas: motores de federación, extensiones de SQL pensadas para NoSQL, bases multi-modelo. Pero ninguna resuelve del todo la necesidad de poder escribir consultas SQL y que estas se ejecuten, sin más, sobre un almacén clave-valor. Un middleware que haga esa traducción de forma transparente sigue siendo una pieza que falta en el rompecabezas.

Precisamente eso es lo que propone este proyecto: construir una capa intermedia que reciba sentencias SQL estándar y las convierta, sobre la marcha, en operaciones compatibles con bases de datos NoSQL de tipo clave-valor. El usuario final escribe su consulta como siempre, por debajo, el sistema se encarga de la transformación semántica y sintáctica necesaria.

Para lograrlo se construyó un parser SQL a medida que descompone cada consulta en piezas manejables, conectores específicos para la base NoSQL elegida, y una interfaz donde el usuario puede escribir su sentencia y ver los resultados sin tener que aprender un lenguaje nuevo.

Más allá de lo técnico, la ventaja es clara: se aprovecha todo el conocimiento acumulado en SQL, que no es poco, dentro de escenarios modernos y escalables, reduciendo tiempos de desarrollo y facilitando la convivencia entre sistemas heterogéneos.

En síntesis, lo que se presenta es un puente entre dos formas de concebir la persistencia de datos. La tesis documenta desde la concepción inicial hasta las pruebas finales, incluyendo los desafíos técnicos encontrados y los ajustes realizados durante el diseño. El trabajo se fundamenta en conceptos de ingeniería de software, teoría de lenguajes de consulta y arquitectura orientada a servicios, aplicados a un proyecto que se desarrolló de forma iterativa mediante sprints.

%%%%%%%%%%%%%%%%%%%%%%%%%%%%%%%%%%%%%%%%%%%
%--------------- Section 1 ---------------%
%%%%%%%%%%%%%%%%%%%%%%%%%%%%%%%%%%%%%%%%%%%
\section{Objetivo general}
\label{chapter01-section01:Objetivo General}
Desarrollar un middleware capaz de traducir consultas SQL en operaciones ejecutables sobre bases de datos NoSQL clave-valor, permitiendo la interoperabilidad entre distintos modelos de almacenamiento y mejorando la eficiencia en el manejo de datos.


%%%%%%%%%%%%%%%%%%%%%%%%%%%%%%%%%%%%%%%%%%%
%--------------- Section 2 ---------------%
%%%%%%%%%%%%%%%%%%%%%%%%%%%%%%%%%%%%%%%%%%%
\section{Objetivos específicos}
\label{chapter01-section02:Objetivos Específicos}
\begin{enumerate}
    \item Analizar las estructuras de gramáticas SQL y los mecanismos de traducción hacia lenguajes NoSQL.
    \item Diseñar la arquitectura del middleware que integre un parser SQL y un conector para bases de datos clave-valor.
    \item Implementar el parser SQL encargado de descomponer las sentencias SQL en componentes manejables.
    \item Desarrollar el conector que traduzca las consultas SQL a operaciones del modelo clave-valor.
    \item Integrar los módulos del sistema dentro de una interfaz funcional que permita ingresar consultas y visualizar resultados.
    \item Evaluar el rendimiento y precisión del middleware mediante pruebas funcionales y de desempeño.
\end{enumerate}


%%%%%%%%%%%%%%%%%%%%%%%%%%%%%%%%%%%%%%%%%%%
%--------------- Section 3 ---------------%
%%%%%%%%%%%%%%%%%%%%%%%%%%%%%%%%%%%%%%%%%%%
\section{Alcance}
\label{chapter01-section03:Alcance}
El alcance del componente comprende todas las fases del ciclo de desarrollo del middleware, desde la investigación inicial hasta la validación final del sistema. Incluye las siguientes etapas:

\begin{itemize}
    \item \textbf{Fase de análisis y diseño:} Revisión de literatura sobre gramáticas SQL y lenguajes NoSQL, definición de la gramática SQL y diseño de la arquitectura general del middleware.
    \item \textbf{Fase de implementación:} Desarrollo del parser SQL y del conector para la base de datos clave-valor (Firebase Realtime Database), asegurando la correcta traducción de consultas y la comunicación entre módulos.
    \item \textbf{Fase de pruebas y evaluación:} Ejecución de pruebas funcionales, medición de usabilidad mediante evaluación heurística y validación de resultados obtenidos frente a las especificaciones del sistema.
\end{itemize}

El componente culmina con la integración de los módulos desarrollados en una interfaz de usuario que permite la interacción directa con el middleware, garantizando la correcta traducción y ejecución de las consultas SQL sobre un entorno NoSQL de tipo clave-valor.


%%%%%%%%%%%%%%%%%%%%%%%%%%%%%%%%%%%%%%%%%%%
%--------------- Section 4 ---------------%
%%%%%%%%%%%%%%%%%%%%%%%%%%%%%%%%%%%%%%%%%%%
\section{Marco teórico}
\label{chapter01-section03:Marco teórico}
\subsection{Antecedentes}

La literatura sobre traducción SQL-NoSQL recoge diversos intentos por resolver un problema común: permitir que aplicaciones con dominio de SQL puedan aprovechar las bases NoSQL sin requerir una curva de aprendizaje extensa.

Namdeo y Suman desarrollaron un middleware en Java denominado \textit{SQL-No-QT} que traduce consultas básicas a MongoDB~\cite{namdeo2022middleware}. Las pruebas realizadas contra Studio 3T mostraron tiempos hasta un 78\% mejores. Por su parte, Dede et al.~\cite{dede2019relational} analizaron el proceso de migración de aplicaciones relacionales a NoSQL mediante \textit{wrappers} que encapsulan la lógica de negocio, evitando la reescritura completa del código.

En el ámbito de la automatización, Queiroz et al.~\cite{queiroz2022amanda} presentaron \textit{AMANDA}, un sistema que migra esquemas y datos de SQL a DGraph con una velocidad 26 veces superior a otras herramientas. Murthy et al.~\cite{murthy2021query} propusieron una plataforma unificada que actúa como middleware plurilingüe, traduciendo comandos genéricos al motor NoSQL correspondiente. Finalmente, Mami et al.~\cite{mami2019survey} realizaron un mapeo exhaustivo de más de 40 enfoques distintos, identificando como problemas principales la pérdida de semántica en relaciones anidadas y la ausencia de un lenguaje intermedio estándar.

\subsection{Fundamentos de bases de datos}

Comprender el funcionamiento de las bases relacionales y NoSQL resulta esencial para diseñar un traductor efectivo. A continuación se presenta un repaso de cada paradigma y de los conceptos que influyeron en las decisiones técnicas del proyecto.

\subsubsection{SQL y el modelo relacional}

El modelo relacional constituye el estándar durante décadas: tablas, columnas, filas y relaciones entre ellas. Las propiedades ACID (Atomicidad, Consistencia, Aislamiento, Durabilidad) garantizan la integridad de los datos incluso ante fallos del sistema~\parencite{melton2002sql}. Sin embargo, la rigidez esquemática (definición previa de esquemas, mantenimiento de claves foráneas) representa un cuello de botella cuando se requiere escalabilidad horizontal o manejo de estructuras variables. Entre los sistemas gestores más representativos se encuentra Microsoft SQL Server, ampliamente documentado en la literatura~\parencite{delaney2013sql}.

\subsubsection{Enfoque NoSQL}

Las bases NoSQL surgieron para resolver las limitaciones de escalabilidad horizontal y flexibilidad de esquema~\parencite{corbellini2016nosql}. A cambio, sacrifican consistencia inmediata (teorema CAP), lo cual resulta aceptable para muchos casos de uso. El desafío radica en que cada motor NoSQL posee su propia API y sintaxis, por lo que el middleware documentado actúa como traductor universal desde SQL.

\subsubsection{Bases de datos clave-valor}

El modelo clave-valor destaca por su simplicidad: almacena pares clave - valor optimizando lecturas y escrituras de baja latencia~\parencite{decandia2007dynamo}.

\paragraph{Firebase Realtime Database:} Servicio de Google que almacena datos como un árbol JSON en la nube. Se sincroniza en tiempo real y ofrece un SDK para Python. Se seleccionó para validar la traducción contra un servicio real en producción~\parencite{google2024firebase}.

\subsection{Lenguajes y herramientas para el análisis de consultas}

El análisis de consultas SQL requiere tokenización, identificación de cláusulas, extracción de valores y manejo de alias. A continuación se describen las herramientas utilizadas.

\subsubsection{Python y librerías de análisis}

Se seleccionó Python por su legibilidad y ecosistema de librerías~\parencite{tiobe2024index, gift2020python}. Para el análisis léxico se utilizó \texttt{sqlparse}, que tokeniza sentencias SQL sin validación semántica completa~\parencite{albrecht2023sqlparse}. Se complementó con el módulo \texttt{re} (expresiones regulares) para capturar patrones complejos como operadores \texttt{LIKE} o condiciones múltiples~\parencite{lutz2021programming}.

\subsubsection{Herramientas de desarrollo}

El flujo de trabajo se apoyó en Visual Studio Code~\parencite{microsoft2023vscode}, Git/GitHub para control de versiones~\parencite{chacon2014progit, github2023platform} y Figma para los prototipos de interfaz~\parencite{figma2023platform}. Los diagramas de arquitectura se generaron con Mermaid, permitiendo un enfoque de diagramación como código.

\subsection{Frameworks}

\subsubsection{Flask}

Microframework de Python utilizado para exponer el middleware como API REST. Su ligereza y la función \texttt{jsonify()} facilitan la serialización de respuestas JSON~\parencite{relan2019flask, grinberg2018flask}.

\subsubsection{React y Next.js}

Para el frontend se utilizó React por su capacidad de construir interfaces reactivas con componentes reutilizables~\parencite{banks2020learning, hoque2020fullstack}. Se integró Next.js para el renderizado híbrido (SSR/CSR) y una estructura de rutas organizada, logrando una interfaz donde el usuario escribe consultas y visualiza resultados de forma inmediata~\parencite{vercel2024docs}.

\subsection{Metodología de desarrollo}

Se adoptó \textbf{Scrum} debido a las incertidumbres iniciales del proyecto: no estaba claro el alcance de traducción SQL soportable ni el comportamiento de Firebase ante filtros complejos~\parencite{schwaber2020scrum}. Los sprints de dos semanas permitieron iterar, validar supuestos y ajustar el rumbo oportunamente~\parencite{sutherland2014scrum}.

\subsection{Pruebas de software}

Para verificar el funcionamiento del middleware se aplicaron dos tipos de pruebas:
\begin{itemize}
    \item \textbf{Pruebas funcionales:} Se verificó que cada tipo de consulta SQL genere la salida NoSQL correcta mediante pruebas de caja negra~\parencite{sommerville2011software}.
    \item \textbf{Pruebas no funcionales:} Se evaluó la usabilidad de la interfaz mediante una evaluación heurística~\parencite{burnstein2003practical}.
\end{itemize}