\abstract{spanish}{El crecimiento exponencial en el volumen de datos ha impulsado la adopción de bases de datos NoSQL, sin embargo, la falta de un lenguaje de consulta estandarizado crea una barrera significativa para desarrolladores formados en el paradigma relacional. Este proyecto aborda dicha problemática mediante el desarrollo de un middleware capaz de traducir sentencias SQL estándar a operaciones nativas para bases de datos documentales en tiempo real. La solución se construyó sobre una arquitectura de capas, desacoplando un frontend educativo en Next.js de un núcleo de procesamiento en Python/Flask. Se diseñó e implementó una ``Estrategia de ejecución híbrida'' (Fetch \& Filter) que permite resolver consultas complejas (como filtros OR y búsquedas ILIKE) que motores como Firebase Realtime Database no soportan nativamente. La validación se realizó mediante casos de prueba funcionales y una evaluación heurística basada en los principios de Nielsen, donde el sistema obtuvo una calificación de severidad 0 en prevención de errores y control de usuario. Los resultados demuestran que la herramienta no solo garantiza la interoperabilidad técnica, sino que reduce significativamente la curva de aprendizaje, sirviendo como un puente pedagógico eficaz entre ambos paradigmas de bases de datos.}{traducción SQL-NoSQL, arquitectura de capas, Next.js, Firebase, estrategia híbrida, evaluación heurística}
 
 \abstract{english}{The exponential growth in data volume has driven the adoption of NoSQL databases, however, the lack of a standardized query language creates a significant barrier for developers trained in the relational paradigm. This project addresses this issue by developing a middleware capable of translating standard SQL statements into native operations for document databases in real-time. The solution was built on a layered architecture, decoupling an educational frontend in Next.js from a processing core in Python/Flask. A ``Hybrid execution strategy'' (Fetch \& Filter) was designed and implemented, enabling the resolution of complex queries (such as OR filters and ILIKE searches) that engines like Firebase Realtime Database do not natively support. Validation was conducted through functional test cases and a heuristic evaluation based on Nielsen's principles, where the system achieved a severity 0 rating in error prevention and user control. The results demonstrate that the tool not only ensures technical interoperability but also significantly reduces the learning curve, serving as an effective pedagogical bridge between both database paradigms.}{SQL-NoSQL translation, layered architecture, Next.js, Firebase, hybrid strategy, heuristic evaluation}