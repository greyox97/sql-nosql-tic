En este anexo se detallan los pasos necesarios para configurar el entorno de desarrollo y desplegar la aplicación localmente, asegurando la reproducibilidad de los resultados presentados en esta tesis.

\section*{Requisitos previos}
\begin{itemize}
    \item \textbf{Python:} Versión 3.8 o superior.
    \item \textbf{Node.js:} Versión 18.x o superior.
    \item \textbf{Cuenta de Google:} Para acceder a Firebase Console.
    \item \textbf{Git:} Para clonar el repositorio.
\end{itemize}

\section*{Configuración de Firebase}

El backend requiere una base de datos Firebase Realtime Database como almacén NoSQL. A continuación se describen los pasos para configurarla:

\subsection*{1. Crear proyecto en Firebase Console}
\begin{enumerate}
    \item Acceder a \texttt{https://console.firebase.google.com/}
    \item Hacer clic en ``Agregar proyecto'' e ingresar un nombre (ej. \texttt{middleware-sql-nosql}).
    \item Desactivar Google Analytics (opcional para este proyecto) y crear el proyecto.
\end{enumerate}

\subsection*{2. Habilitar Realtime Database}
\begin{enumerate}
    \item En el panel lateral, seleccionar ``Realtime Database'' bajo la sección ``Compilación''.
    \item Hacer clic en ``Crear base de datos''.
    \item Seleccionar la ubicación del servidor (ej. \texttt{europe-west1}).
    \item Iniciar en ``modo de prueba'' para desarrollo (permite lectura/escritura sin autenticación por 30 días).
\end{enumerate}

\subsection*{3. Obtener credenciales de servicio}
Para que el backend Python pueda autenticarse con Firebase:
\begin{enumerate}
    \item Ir a ``Configuración del proyecto'' (icono de engranaje) $\rightarrow$ ``Cuentas de servicio''.
    \item Hacer clic en ``Generar nueva clave privada''.
    \item Descargar el archivo JSON y renombrarlo a \texttt{credenciales-firebase.json}.
    \item Colocar este archivo en el directorio \texttt{backend/database/}.
\end{enumerate}

\textbf{Nota de seguridad:} El archivo de credenciales contiene información sensible. Nunca debe subirse a repositorios públicos. El archivo \texttt{.gitignore} del proyecto ya incluye esta exclusión.

\section*{Configuración del backend (Flask)}

El backend fue desarrollado manualmente sin utilizar generadores de código. La estructura modular se creó archivo por archivo siguiendo el patrón MVC.

\subsection*{1. Clonar el repositorio}
\begin{lstlisting}[language=bash]
git clone https://github.com/greyox97/sql-nosql-tic.git
cd sql-nosql-tic
\end{lstlisting}

\subsection*{2. Crear entorno virtual e instalar dependencias}
\begin{lstlisting}[language=bash]
cd backend
python -m venv venv
# En Windows:
venv\Scripts\activate
# En Linux/Mac:
source venv/bin/activate

pip install -r requirements.txt
\end{lstlisting}

Las dependencias principales incluyen:
\begin{itemize}
    \item \texttt{flask}: Framework web ligero.
    \item \texttt{flask-cors}: Manejo de CORS para comunicación con el frontend.
    \item \texttt{firebase-admin}: SDK oficial de Firebase para Python.
    \item \texttt{sqlparse}: Parser de sentencias SQL.
\end{itemize}

\subsection*{3. Configurar variables de entorno}
Crear un archivo \texttt{.env} en el directorio \texttt{backend/} con el siguiente contenido:
\begin{lstlisting}[language=bash]
FIREBASE_CREDENTIALS=database/credenciales-firebase.json
FIREBASE_DB_URL=https://tu-proyecto-default-rtdb.europe-west1.firebasedatabase.app/
\end{lstlisting}

Reemplazar \texttt{tu-proyecto} por el nombre real del proyecto Firebase.

\section*{Configuración del frontend (Next.js)}

\subsection*{1. Instalar dependencias}
Navegar al directorio del frontend e instalar las dependencias:
\begin{lstlisting}[language=bash]
cd frontend
npm install
\end{lstlisting}

\textbf{Nota técnica:} El proyecto fue inicialmente creado utilizando:
\begin{lstlisting}[language=bash]
npx create-next-app@latest frontend --typescript --tailwind --eslint --app --src-dir
\end{lstlisting}

Este comando configuró automáticamente TypeScript, Tailwind CSS, ESLint y el App Router. Para la replicación del proyecto, este paso no es necesario ya que la estructura generada está incluida en el repositorio.

\section*{Ejecución del sistema}

Una vez configurados ambos componentes, iniciar los servidores de desarrollo en dos terminales separadas:

\subsection*{Terminal 1 - Backend (Puerto 5000)}
\begin{lstlisting}[language=bash]
cd backend
python app.py
\end{lstlisting}

\subsection*{Terminal 2 - Frontend (Puerto 3000)}
\begin{lstlisting}[language=bash]
cd frontend
npm run dev
\end{lstlisting}

La aplicación estará disponible en \texttt{http://localhost:3000}. El frontend se comunicará automáticamente con el backend en el puerto 5000 gracias a la configuración de CORS implementada.

\section*{Verificación del despliegue}

Para confirmar que el sistema funciona correctamente:
\begin{enumerate}
    \item Abrir \texttt{http://localhost:3000} en el navegador.
    \item Ingresar una consulta de prueba: \texttt{SELECT * FROM northwind}
    \item Verificar que se muestren resultados en la tabla y la traducción NoSQL en el panel derecho.
\end{enumerate}