En este anexo se detallan los pasos necesarios para desplegar la aplicación en un entorno local, asegurando la reproducibilidad de los resultados presentados en esta tesis.
 
 \section*{Requisitos previos}
 \begin{itemize}
     \item \textbf{Python:} Versión 3.8 o superior.
     \item \textbf{Node.js:} Versión 18.x o superior.
     \item \textbf{Firebase:} Proyecto activo en Firebase Console con Realtime Database habilitado.
 \end{itemize}
 
 \section*{Pasos de instalación}
 
 \subsection*{1. Clonar el repositorio}
 \begin{lstlisting}[language=bash]
 git clone https://github.com/greyox97/sql-nosql-tic.git
 cd sql-nosql-tic
 \end{lstlisting}
 
 \subsection*{2. Configurar el Backend (Flask)}
 Navegar al directorio raíz del backend y crear un entorno virtual:
 \begin{lstlisting}[language=bash]
 cd backend
 python -m venv venv
 source venv/bin/activate  # En Windows: venv\Scripts\activate
 pip install -r requirements.txt
 \end{lstlisting}
 
 Configurar las variables de entorno creando un archivo \texttt{.env} en la raíz del backend:
 \begin{lstlisting}[language=bash]
 FIREBASE_CREDENTIALS=ruta/a/tus/credenciales.json
 FIREBASE_DB_URL=https://tu-proyecto.firebaseio.com/
 \end{lstlisting}
 
 \subsection*{3. Configurar el Frontend (Next.js)}
 Navegar al directorio del frontend e instalar dependencias:
 \begin{lstlisting}[language=bash]
 cd ../frontend
 npm install
 \end{lstlisting}
 
 \subsection*{4. Ejecución}
 Iniciar los servidores de desarrollo en dos terminales separadas:
 
 \textbf{Terminal 1 (Backend):}
 \begin{lstlisting}[language=bash]
 flask run --debug --port 5000
 \end{lstlisting}
 
 \textbf{Terminal 2 (Frontend):}
 \begin{lstlisting}[language=bash]
 npm run dev
 \end{lstlisting}
 
 La aplicación estará disponible en \texttt{http://localhost:3000}.